\documentclass{article}
\usepackage{amsmath}
\usepackage{graphicx}
\usepackage{amssymb}

\title{Nepal TST25 Solutions}
\author{Prakrit Gajurel}
\date{}

\begin{document}

\maketitle

\noindent\rule{12cm}{0.4pt} \\
\noindent \textbf{Problem 2} (Kritesh Dhakal, Nepal)
\\
Find all integers $n$ such that if
\begin{align*}
1 = d_1 < d_2 <\cdots < d_{k-1} < d_k = n
\end{align*}
are the divisors of $n$, then the sequence
\begin{align*}
d_2-d_1, d_3-d_2, \dots, d_k-d_{k-1}
\end{align*}
forms a permutation of an arithmetic progression. \\
\noindent\rule{12cm}{0.4pt} \\
\noindent \textit{Solution. } The only $n$ that satisfies the condition(except $n=1$) is $n=p$, $n=p^2$, $n=pq$, and n=10 for some primes $p$ and $q$.
\\ \\ \\
$\boxed{\textbf{CLAIM}: d_k-d_{k-1} \text{ is always the largest term in the arithmetic progression.}}$ \\ \\
$\textit{proof. }$ Let $p_0$ be the smallest prime dividing $n$, then,
\begin{align*}
   d_k-d_{k-1} \ge n-\frac{n}{p} \ge n-\frac{n}{2} = \frac{n}{2} \ge d_i > d_i - d_{i-1}
\end{align*}
for some $i \le k-1$, which completes the proof for the claim. $\square$
\\ \\
$\boxed{\textbf{CLAIM}: \text{No } k \ge 5 \text{ works.}}$ \\ \\
$\textit{proof. }$ Assume not, for the sake of contradiction. Then, we notice,
\begin{center}
    $\Sigma \text{ terms} = (d_2-d_1) + (d_3-d_2) + \cdots + (d_k - d_{k-1})$ \\
    $\Sigma \text{ terms} = d_k - d_1$ \\
    $\Sigma \text{ terms} = n - 1$ \\
    $\frac{k-1}{2}\left[  (d_2-d_1) + n - \frac{n}{p_0}\right] = n-1$
\end{center}
Now, notice:
\begin{center}
    $\frac{k-1}{2}\left[  (d_2-d_1) + n - \frac{n}{p_0}\right] > \frac{k-1}{2}\left[n - \frac{n}{p_0}\right] \ge \frac{k-1}{2}\left[n - \frac{n}{2}\right] = \frac{k-1}{4}(n)$
\end{center}
Giving us that $\frac{k-1}{4} < n-1$.
However, if k=5, we get $n<n-1$ which is obviously a contradiction, and thus we get that no $k\ge5$ works. $\square$

\end{document}
