\documentclass{article}
\usepackage{amsmath}
\usepackage{graphicx}
\usepackage{amssymb}

\title{Nepal TST25 Solutions}
\author{Prakrit Gajurel}
\date{}

\begin{document}

\maketitle

\noindent\rule{12cm}{0.4pt} \\
\noindent \textbf{Problem 2} (Kritesh Dhakal, Nepal)
\\
Find all integers $n$ such that if
\begin{align*}
1 = d_1 < d_2 <\cdots < d_{k-1} < d_k = n
\end{align*}
are the divisors of $n$, then the sequence
\begin{align*}
d_2-d_1, d_3-d_2, \dots, d_k-d_{k-1}
\end{align*}
forms a permutation of an arithmetic progression. \\
\noindent\rule{12cm}{0.4pt} \\
\noindent \textit{Solution. } We claim the only $n$ that satisfies the condition(except $n=1$) is $n=p$, $n=p^2$, $n=pq$, and n=10 for some primes $p$ and $q$.
\\ \\ \\
$\boxed{\textbf{CLAIM}: d_k-d_{k-1} \text{ is always the largest term in the arithmetic progression.}}$ \\ \\
$\textit{proof. }$ Let $p_0$ be the smallest prime dividing $n$, then,
\begin{align*}
   d_k-d_{k-1} \ge n-\frac{n}{p} \ge n-\frac{n}{2} = \frac{n}{2} \ge d_i > d_i - d_{i-1}
\end{align*}
for some $i \le k-1$, which completes the proof for the claim. $\square$
\\ \\
$\boxed{\textbf{CLAIM}: \text{No } k \ge 5 \text{ works.}}$ \\ \\
$\textit{proof. }$ Assume not, for the sake of contradiction. Then, we notice,
\begin{center}
    $\Sigma \text{ terms} = (d_2-d_1) + (d_3-d_2) + \cdots + (d_k - d_{k-1})$ \\
    $\Sigma \text{ terms} = d_k - d_1$ \\
    $\Sigma \text{ terms} = n - 1$ \\
    $\frac{k-1}{2}\left[  (d_2-d_1) + n - \frac{n}{p_0}\right] = n-1$
\end{center}
Now, notice:
\begin{center}
    $\frac{k-1}{2}\left[  (d_2-d_1) + n - \frac{n}{p_0}\right] > \frac{k-1}{2}\left[n - \frac{n}{p_0}\right] \ge \frac{k-1}{2}\left[n - \frac{n}{2}\right] = \frac{k-1}{4}(n)$
\end{center}
Giving us that $\frac{k-1}{4} < n-1$.
However, if k=5, we get $n<n-1$ which is obviously a contradiction, and thus we get that no $k\ge5$ works. $\square$
 \\ \\ \\
 If $n$ has a maximum of 4 factors, then we have four cases.
 \\ \\
 \textbf{Case 1. } $n=p$ for some prime $p$ always works as one term is always in an A.P.
 \\ \\
 \textbf{Case 2. } $n=p^2$ for some prime $p$ always works as two terms are always in an A.P.
 \\ \\
 \textbf{Case 3. } $n=p^3$ for some prime $p$.
 \\ 
 \indent Here, the factors of $n$ come out to be $1, p, p^2, p^3$, giving us the permutation of the A.P as $p-1, p^2-p, p^3-p^2$.
 \\ For them to be in an A.P, the following must hold:
 \begin{center}
    $(p^3-p^2) - (p^2-p) = (p^2-p) - (p-1)$ \\
    $(p^3-p^2) = 2(p^2-p) - (p-1)$ \\
    $p^3-p^2 = 2p^2-2p-p+1$ \\
    $p^3-p^2=2p^2-3p+1$ \\
    For this to hold, since $p$ divides the LHS, it also must divide the RHS. But, $p|2p^2-3p+1 \implies p|1$ which is a contradiction, thus $n=p^3$ does not work. The other permutation of the A.P gives a similar result.
 \end{center}
 $\textbf{Case 4. } n=pq$ for some primes $p$ and $q$.
 \\
 Without loss of generality, let $q>p$. Then, the factors would be $1, p, q, pq$ with the permutation of the A.P being $p-1, q-p, pq-q$ which gives us two subcases.
 \\ \\ $\textbf{Subcase 4.1. }$ The A.P is $p-1, q-p, pq-q$ in ascending order.
 \begin{center}
    $(pq-q)-(q-p) = (q-p)-(p-1)$ \\
    $2(q-p) - (p-1) = pq-q$ \\
    $3q-pq-3p=-1$ \\
    $(q+3)(p-3)=-8 \implies q=5, p=2 \implies n=10$ which works upon checking.
 \end{center}
 \indent $\textbf{Subcase 4.2.}$
 \begin{center}
    $(pq-q)-(p-1)=(p-1)-(q-p)$ \\
    $pq-3p=-2$ \\
    $p(q-3)=-2 \implies p=q=2$, contradicting $q>p$ thus this case doesn't work.
 \end{center}
 Thus, we have shown that $n=p$, $p^2$, $pq$, $10$ for all primes $p$ and $q$ are the only solutions to $n$ apart from $n=1$. $\blacksquare$
\end{document}
